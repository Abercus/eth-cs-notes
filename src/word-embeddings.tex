\documentclass[12pt]{article}

\usepackage{notes}

\begin{document}

\lhead{Spring 2019}
\chead{Word Embeddings}
\rhead{\today}

\title{Word Embeddings}
\author{Doruk Çetin}
%\maketitle

%\setcounter{tocdepth}{1}
%\tableofcontents
%\pagebreak

\section{NLU: Introduction}
\par \textbf{Synonymy}: equivalence of near-equivalence in meaning, \textbf{polysemy}: multitude of meanings
\par Regarding polysemy, disambiguating the whole corpus would be cumbersome. Even more problematically, it is unclear what the meanings are anyway.
\par \textbf{Raw dot-product}, has a problem as a similarity metric: it favors long vectors. The dot product is higher if a vector is longer, with higher values in each dimension. The simplest way to modify the dot product to normalize for the vector length is to divide the dot product by the lengths of each of the two vectors. This normalized dot product turns out to be the same as the \textbf{cosine} of the angle between the two vectors.
\par \textbf{Pointwise mutual information} (is one of the most important concepts in NLP. It is a measure of how often two events x and y occur, compared with what we would expect if they were independent:
$I(x,y) = \log_2 \frac{P(x,y)}{P(x)P(y)}$. It is more common to use \textbf{Positive PMI} (called PPMI) which replaces all negative PMI values with zero.
\par Distribution similarity: Compare words by comparing their distributions over context words, e.g. KL divergence, Jensen-Shannon divergence
\par Two words have \textbf{first-order cooccurrence} (sometimes called \textbf{syntagmatic association}) if they are typically nearby each other. Thus wrote is a first-order associate of book or poem. Two words have \textbf{second-order cooccurrence} (sometimes called \textbf{paradigmatic association}) if they have similar neighbors. Thus wrote is a second-order associate of words like said or remarked.
\par Word embeddings are based on the distributional assumption that words appearing within similar context possess similar meaning.
\par Word2Vec tends to embed both syntactical and semantic information and it is very effective for the compositionality.

\section{ML4H: Motivation}
\par Discrete representations: use a taxonomy that has hypernyms (is-a) relationships and synonym sets (e.g. WordNet).
\\
Problems with discrete representations:
\ulb
\item Missing new words - hard to keep up with the evaluation of the language.
\item Requires human labor to create and adapt.
\item Subjective.
\item Hard to compute accurate word similarity.
\item Dimensionality problems (one-hot encoding).
\ule
Problems with simple co-occurrence vectors:
\ulb
\item Increase in size with vocabulary. Very high dimensional: require a lot of storage.
\item Subsequent classification models have sparsity issues. Less robust models as a result.
\item Dimensionality reduction with SVD: Computational cost scales quadratically for $n*m$ matrix: $O(mn2)$, hard to incorporate new words or documents
\ule
\textbf{Why use embeddings?}
\ulb
\item Reduce dimensionality of representation.
\item Encodes similarity information, useful for other tasks.
\item Learn representations of entities (words) as well as relationships between them.
\ule
 
\section{NLU: skip-gram architecture}
\par The word2vec family of models, including skip-gram and CBOW, is a popular efficient way to compute dense embeddings.
\par The \textbf{intuition of word2vec} is that instead of counting how often each word w occurs near, say, apricot, we'll instead train a classifier on a binary prediction task: "Is word w likely to show up near apricot?" We don't actually care about this prediction task; instead we'll take the learned classifier weights as the word embeddings. The revolutionary intuition here is that we can just use running text as implicitly supervised training data for such a classifier.
Skip-gram intuition:
\ulb
\item Treat the target word and a neighboring context word as positive examples.
\item Randomly sample other words in the lexicon to get negative samples
\item Use logistic regression to train a classifier to distinguish those two cases
\item Use the regression weights as the embeddings
\ule
\par Softmax in log-bilinear model needs the computation of partition function. Noise contrastive estimation or negative sampling allows an alternative.
\par The noise words are chosen according to their weighted unigram frequency $p_{\alpha}w = \frac{count(w)^{\alpha}}{\sum_{w'}count(w')^{\alpha}}$, where $\alpha$ is a weight.
\par For $k=1$ (no oversampling) and $p_n(w_i, w_j)=p(w_i)p(w_j)$ negative sampling yields pointwise mutual information.
\par word2vec vs count-based methods:
\ulb
\pro scale with corpus size; capture complex patterns beyond word similarity
\con slower than occurrence count based model; efficient usage of statistics
\ule

\section{etc: CBOW vs. Skip-Gram}
\par \href{https://stackoverflow.com/questions/38287772/cbow-v-s-skip-gram-why-invert-context-and-target-words#}{\textbf{Reference website}}: In the "skip-gram" mode alternative to "CBOW", rather than averaging the context words, each is used as a pairwise training example. That is, in place of one CBOW example such as [predict 'ate' from average('The', 'cat', 'the', 'mouse')], the network is presented with four skip-gram examples [predict 'ate' from 'The'], [predict 'ate' from 'cat'], [predict 'ate' from 'the'], [predict 'ate' from 'mouse']. (The same random window-reduction occurs, so half the time that would just be two examples, of the nearest words.)
\ulb
\item Skip-gram: works well with small amount of the training data, represents well even rare words or phrases.
\item CBOW: several times faster to train than the skip-gram, slightly better accuracy for the frequent words
\ule

\section{ML4H: Word2Vec}
\textbf{Word2Vec}
\ulb
\item Train a classifier on a binary prediction task of words occurring in the neighbourhoods of other words, take the learned classifier weights as the word embeddings.
\item Faster and can easily incorporate a new sentence/document or add a word to the vocabulary.
\item \textbf{Continuous Bag-of-Words (CBoW) model}: predict center word from sum of surrounding word vectors (see NLU notes)
\item \textbf{Skip-gram model}: predicting surrounding single words from center word (see NLU notes). Take the target word and a neighboring context word as positive examples, randomly sample other words in the lexicon to get negative samples.
\item Normalized dot-product gives \textbf{cosine similarity}. Normalize similarities to get probabilities: $P(+  \given  t,c) = 1 / (1 + e^{-sim(t,c)})$
\item This means we maximise the overlap (via dot product) between a word and the context it appeared in. By transitivity, any other word with a similar context will have a large overlap with the original word. For example, jumps $\sim$ leaps because their context vectors are similar.
\ule
\textbf{FastText}
\ulb
\item Extension of word2vec model, treats each word as composed of character n-grams.
\item Can generate embeddings for OOV words, generate better embeddings than word2vec for rare words.
\item Requires bigger memory and longer training times.
\ule

\section{CIL: Skip-gram and negative sampling}
\ulb
\item \textbf{Skip-gram model:} predict context word, given active word
\item Distributional semantics model = distribution of co-occurring words determines lexical semantics.
\item An alternative to using windows would be to use words within same sentence.
\item log-bilinear model where $w \mapsto (\mathbf{x}_w, b_w)\in\R^{d+1}$:
\[ \log p_{\theta}(w\given w') = \left\langle \mathbf{x}_w, \mathbf{x}_{w'} \right\rangle + b_w + const \]
\item log-likelihood of the basic model becomes
\[ \mathcal{L}(\theta;\mathbf{w}) = \sum_{t=1}^T \sum_{\Delta\in\mathcal{I}} \left[
\left\langle \mathbf{x}_{w^{(t+\Delta)}}, \mathbf{x}_{w^{(t)}} \right\rangle + b_{w^{(t+\Delta)}}
-\log \sum_{v\in\mathcal{V}} \exp\left[ \left\langle \mathbf{x}_{v}, \mathbf{x}_{w^{(t)}} \right\rangle + b_{v} \right]
\right] \]
\item \textbf{Context Vectors:} using the same vectors for context and active words would be imposing and implicit constraint (bi-linearity), so we can use different vectors for inputs and output embeddings. Model dimensionality grows larger but this adds more flexibility to the model.
\item \textbf{Negative sampling:} simplified version of noise contrastive estimation where the aim is to reduce estimation to binary classification. Negative sampling is done through sampling "random" ($P(w_j)^{\alpha}, \text{ e.g. } \alpha=3/4$) context words. Exponent (smaller than one) of the unigram probability dampens frequent words. We move to logistic function from softmax, that needs the computation of partition function with large cardinality.
\item log-likelihood of the model with context vectors and negative sampling:
\[ \mathcal{L}(\theta) = \sum_{(i,j)\in\Delta^+} \log\sigma\left(\left\langle \mathbf{x}_i,\mathbf{y}_j \right\rangle\right) + \sum_{(i,j)\in\Delta^-} \log\sigma\left(-\left\langle \mathbf{x}_i,\mathbf{y}_j \right\rangle\right) \]
\item In a way, what skip-gram model with negative sampling does is essentially performing a low-rank decomposition of the pointwise mutual information matrix. Negative sampling idea which was a bit heuristic actually seems to be doing something quite reasonable.
\ule

\section{CIL: GloVe}
\par Using the same notation for output vocabulary $\mathcal{V}$ and input vocabulary $\mathcal{C}$, co-occurence matrix $\mathcal{N} = (n_{ij})\in \mathbb{N}^{|\mathcal{V}|\cdot|\mathcal{C}|}$ where $n_{ij}$ = \# occurences of $w_i\in\mathcal{V}$ in context of $w_j\in\mathcal{C}$. It is a sparse matrix which can be computed in one pass over the corpus.
\par \textbf{GloVe objective} is the weighted least squares fit of log-counts:
\[ \mathcal{H}(\theta;\mathbf{N}) = \sum_{i,j} f(n_{ij}) \left( \log n_{ij} - \log \widetilde{p}_{\theta} (w_i\given w_j) \right)^2 \]
with unnormalized distribution
\[ \widetilde{p}_{\theta} (w_i\given w_j) = \exp\left[ \left\langle \mathbf{x}_i,\mathbf{y}_j \right\rangle + b_i + c_j \right] \]
and weighting function $f(n) = \min \left\{ 1, \left( \frac{n}{n_{\max}} \right)^{\alpha}  \right\} \text{ e.g. } \alpha = \frac{3}{4}  $
\par GloVe with $f\coloneqq 1$ solves a matrix factorization problem of the log-count matrix.
\par Advantage of the GloVe is that we model unnormalized probabilities which does not need the computation of the partition function.
\par Optimization through SGD: full gradient is often too expensive to compute for $\mathcal{H}(\theta;\mathbf{N})$, instead use stochastic optimization. We sample $(i,j)$ such that $n_{ij} > 0$ uniformly at random. While the inner product is not big enough (i.e. w.r.t. log-counts) we move the two vectors closer to each other.
\[ \mathbf{x}_i^{\text{new}} \leftarrow \mathbf{x}_i + 2\eta f(n_{ij})(\log n_{ij} - \langle \mathbf{x}_i , \mathbf{y_j} \rangle) \mathbf{y}_j \]
\[ \mathbf{y}_j^{\text{new}} \leftarrow \mathbf{y}_j + 2\eta f(n_{ij})(\log n_{ij} - \langle \mathbf{x}_i , \mathbf{y_j} \rangle) \mathbf{x}_i \]

\section{ML4H: Semantic Relationships}
\par Combining embeddings with prior knowledge: from analogical reasoning, abstract relationships were translations in the embedded space. Take this idea and extend the concept of context to include "appears in a relationship with" alongside "appears in a sentence with" and represent these new context-relationships as arbitrary affine transformations (basically, matrices).
\par Enforcing similarity: define an energy function $\mathcal{E} (S,R,T)$, energy is low if S is related to T through R is true (R is often non-symmetric). An example  energy function is $$\mathcal{E} (S,R,T \given \theta) = -\frac{\mathbf{v}_T \cdot G_R \mathbf{c}_S}{\snorm{\mathbf{v}_T}\snorm{G_R \mathbf{c}_S}}$$

\section{etc: Practicalities}
\par "Off-task" data helps due to shared semantic information.
\par Although pre-trained word vectors work very well in many practical downstream tasks, in some settings it's best to continue to learn (i.e. 'retrain') the word vectors as parameters of our neural network. Explain why retraining the word vectors may hurt our model if our dataset for the specific task is too small: See TV/television/telly example. Word vectors in training data move around; word vectors not in training data don't move around. Destroys structure of word vector space. Could also phrase as an overfitting or generalization problem.
\par Evaluating word vectors:
\ulb
\item Intrinsic: Word Vector Analogies ; Word vector distances and their correlation with human judgments.
\item Extrinsic: Name entity recognitions: finding a person, location or organization and so on.
\ule

\end{document}